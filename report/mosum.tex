\section{Ordinary Least Squares Moving Sum (OLS-MOSUM) Test}
\label{sec:mosum}

Tests from the generalized fluctuation test framework \cite{kuan_hornik}
is one of the most important classes of tests on structural change. This class
of tests include, in particular, Moving Sum (MOSUM) test. In the paper from
2003 \cite{strucchange}, Zeileis et al. describe the underlying theory and
implementation of the \texttt{strucchange}
package for the R programming language, which include multiple tools for
detection of structural changes in linear regression relationships.

\subsection{The model}
\label{sec:mosum}
In \texttt{strucchange} \cite{strucchange}, a standard linear regression model is considered:
\[
y_{i}=x_{i}^{\top} \beta_{i}+u_{i} \quad(i=1, \ldots, n)
\]
where:
\begin{itemize}
\item $x_i = (1,x_{i2}, x_{i3}, ..., x_{ik})^T \in \mathbb{R}^k$
\item $y_i \in \mathbb{R}$
\item $\beta_i \in \mathbb{R}^{k} $ is the
  regression coefficient vector
\item $u_i \in \mathbb{R}$ is the error term, that independently and identically
  distributed with mean $\mu = 0$ and variance $\sigma^2$.
\end{itemize}
We can then test for structural change by testing the null hypothesis:
\[
H_0:\quad \beta_i = \beta_0
\]
which states that there is no structural change present in the time series in
the time series. While the alternative is that the coefficient vector $\beta$
varies over time.


\subsection{Empirical fluctuation process (OLS-MOSUM)}
\subsection{Significance testing}
