\documentclass[main.tex]{subfiles}

\begin{document}
\chapter{Estimation of Breakpoints}
\label{chap:estimation_of_breakpoints}
In the paper from 2003 \cite{bai_perron}, Jushan Bai and Pierre Perron provide an
efficient algorithm for time series breakpoint estimation. It uses a dynamic-programming
approach and requires $O(T^2)$ least-squares operations for any number of breakpoints.


\section{The Model}
\label{sec:breakpoints_the_model}

\section{Recursive Residuals}
\label{sec:recursive_residuals}

\section{The Triangular Matrix of Sums of Squared Residuals}
\label{sec:triangular_matrix}
\subsection{Reductions}
\subsection{Computation of the triangular matrix}

\section{The dynamic programming algorithm}
\label{sec:the_dynamic_programming_algorithm}

	
%% \subsection{Bayesian Information Criterion}
%% \label{subsec:bayesian_information_criterion}

%% \subsection{Calculation of confidence intervals for the break dates}
%% \label{subsec:confidence_intervals}

\biblio
\end{document}
