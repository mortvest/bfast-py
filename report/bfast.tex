\documentclass[main.tex]{subfiles}

\begin{document}
\chapter{BFAST}
\label{chap:bfast}
In paper from 2010, Verbesselt et. al \cite{bfast} outline a generic change detection approach
that combines iterative decomposition into trend, seasonal and remainder components and detection
and characterizing of breakpoints within a time series. 

\section{The model}
\label{sec:bfast_the_model}
For BFAST, we consider a following data model:
\[
Y_t = T_t + S_t + e_t, \quad t = 1,...,n
\]
where:
\begin{itemize}
\item $Y_t \in \mathbb{R}$ is the observation at time $t$
\item $T_t \in \mathbb{R}$ is the trend component at time $t$
\item $S_t \in \mathbb{R}$ is the seasonal component at time $t$
\item $e_t \in \mathbb{R}$ is the remainder component at time $t$
\item $n$ is the number of observations in time series
\end{itemize}
For more information about the time series components, please refer to Chapter
\ref{chap:stl}.
\subsection{The Trend Component}
\label{subsec:trend}
We assume that $T_t$ is piecewise linear and has breakpoints $t_1^*,\hdots, t_m^*$,
where $m$ is the number of breakpoints and set $T_0^* = 0$. Then, the trend component
can be described as
\[
T_t = \alpha_j + \beta_j t \quad \text{for}\quad t^*_{j-1}<t\leq t_j^*
\]
where $j$ is the number of the next breakpoint, i.e. $j = 1,...,m$. One way, in
which we can characterize the abrupt changes in the trend component is by
calculating the magnitude of the change:
\[
\operatorname{Magnitude}(j) = T_{j-1} - T_{j} = \alpha_{j-1} + \beta_{j-1} t -
(\alpha_j + \beta_j t) = (\alpha_{j-1} - \alpha_j) + (\beta_{j-1} - \beta_j)t
\]
where $j = 1,...,m$.

\subsection{The Seasonal Component (Dummy Model)}
\label{subsec:seasonal_dummy}
The breakpoints in the seasonal component can occur at different times than the
breaks in the trend component. We define the seasonal breakpoints:


\subsection{The Seasonal Component (Harmonic Model)}
\label{subsec:seasonal_harm}
Alternatively, we can use the Harmonic model from the BFAST0n algorithm. For
more information, please refer to Chapter \ref{chap:bfast0n}.



\section{Steps of the Algorithm}
\label{sec:bfast_algorithm_steps}
BFAST is an iterative algorithm that for $i = 1 \hdots \operatorname{max}_\text{iter}$,
takes following steps:
\begin{enumerate}
\item 
\end{enumerate}

\section{Parameters and Their Values}
\label{sec:bfast_params}
BFAST algorithm has following paramers
\begin{itemize}
\item $s$. Period of seasonality.
\item $0<h<1$. Minimal segment length for the Breakpoint estimation and
  bandwidth parameter for the OLS-MOSUM test. For more information please refer
  to Chapters \ref{chap:mosum} and \ref{chap:breakpoints} respectively.
  The default value is $0.15$.
\item \texttt{seasonal}. Type of model to be fitted. Should be either
  ``harmonic'', ``dummy'' or ``none''. The later means that no seasonal model
  would be fitted.
\item \texttt{max\_iter}. Maximal number of iterations of the main loop. The
  default value is $10$. In practice 1 or 2 iterations is normally enough.
\item \texttt{max\_breaks}. Upper limit on the number of breakpoints that would be found by
  the breakpoint estimation algorithm. The default value is $None$, i.e. no upper limit.
  For more information, please refer to Chapter \ref{chap:breakpoints}.
\item $\alpha$. The confidence interval for the OLS-MOSUM test. If the returned
  p value is below that threshold - we consider that there is structural change
  and proceed with the linear regression. For more information, please refer to
  the Chapter \ref{chap:breakpoints}. The default value is $0.05$. 
\end{itemize}

\biblio
\end{document}
