\documentclass[main.tex]{subfiles}

\begin{document}
\chapter{BFAST}
\label{chap:bfast}
In paper from 2010, Verbesselt et. al \cite{bfast} outline a generic change detection approach
that combines iterative decomposition into trend, seasonal and noise components and detection
and characterizing of breakpoints within a time series. 

\section{The model}
\label{sec:bfast_the_model}
In BFAST, the Following data model is considered:
\[
Y_t = T_t + S_t + e_t, \quad t = 1,...,n
\]
where:
\begin{itemize}
\item $Y_t \in \mathbb{R}$ is the observation at time $t$
\item $T_t \in \mathbb{R}$ is the trend component at time $t$
\item $S_t \in \mathbb{R}$ is the seasonal component at time $t$
\item $e_t \in \mathbb{R}$ is the remainder component at time $t$
\item $n$ is the number of observations in time series
\end{itemize}
For more information on the time series components, please refer to Chapter \ref{chap:stl}.\\\\
We assume that $T_t$ is piecewise linear and has breakpoints $t_1^*,\hdots, t_m^*$, and set
$T_0^* = 0$. 



that is based on application of STL decomposition \ref{chap:stl} in order to get the initial




%% in which the BFAST approach differs from the similar methods is that it uses the
%% OLS-MOSUM statistical test to determine whether the breakpoints are taking place in the seasonal and
%% trend components of the time series before committing to the costly
%% dynamic-programming algorithm that determines the location and quality of the
%% breakpoints. In the R implementation, Verbeselt et al.\cite{bfast-github} use
%% the \texttt{strucchange} package, originally by Zelis et. al \cite{strucchange}.

\section{Algorithm Overview}
\label{sec:bfast_algorithm_overview}
\section{Algorithm Steps}
\label{sec:algorithm_steps}
\section{Parameters and Their Values}
\label{sec:bfast_params}

\biblio
\end{document}
