\documentclass[main.tex]{subfiles}

\begin{document}
\chapter{BFAST}
\label{chap:bfast}
In paper from 2010, Verbesselt et. al \cite{bfast} outline a generic change detection approach
that combines iterative decomposition into trend, seasonal and remainder components and detection
and characterizing of breakpoints within a time series. 

\section{The model}
\label{sec:bfast_the_model}
For BFAST, we consider a following data model:
\[
Y_t = T_t + S_t + e_t, \quad t = 1,...,n
\]
where:
\begin{itemize}
\item $Y_t \in \mathbb{R}$ is the observation at time $t$
\item $T_t \in \mathbb{R}$ is the trend component at time $t$
\item $S_t \in \mathbb{R}$ is the seasonal component at time $t$
\item $e_t \in \mathbb{R}$ is the remainder component at time $t$
\item $n$ is the number of observations in time series
\end{itemize}
For more information about the time series components, please refer to Chapter
\ref{chap:stl}.
\subsection{The Trend Component}
\label{subsec:trend}
We assume that $T_t$ is piecewise linear and has breakpoints $t_1^*,\hdots, t_m^*$,
where $m$ is the number of breakpoints in the trend component and set $t_0^* = 0$.
Then, the trend component can be described as
\[
T_t = \alpha_j + \beta_j t \quad \text{for}\quad t^*_{j-1}<t\leq t_j^*
\]
where $j$ is the number of the next breakpoint, i.e. $j = 1,...,m$. One way, in
which we can characterize the abrupt changes in the trend component is by
calculating the magnitude of the change:
\[
\operatorname{Magnitude}(j) = T_{j-1} - T_{j} = \alpha_{j-1} + \beta_{j-1} t -
(\alpha_j + \beta_j t) = (\alpha_{j-1} - \alpha_j) + (\beta_{j-1} - \beta_j)t
\]
where $j = 1,...,m$.

\subsection{The Seasonal Component}
\label{subsec:seasonal}
The breakpoints in the seasonal component can occur at different times than the
breaks in the trend component. Let
$t_1^{\#},\hdots, t_p^{\#}$ be the breakpoints in the seasonal component,
where $p$ is the number of breakpoints and $t_0^{\#} = 0$.

\subsection{Dummy Model}
For $t_{j-1}^{\#} < t \leq t_j^{\#}$, the seasonal term can be expressed as:
\[
S_t = \sum_{i=1}^{s-1}\gamma_{i,j}(d_{t, i} - d_{t, 0})
\]
where
\begin{itemize}
\item $s$ is the period of seasonality (the number of observations per year)
\item $\gamma_{i,j}$  is the influence of season $i$ that we are going to estimate
with linear regression. 
\item $d_{t,i}$ is the dummy variable \cite{makridakis}, for which it holds that
  $d_{t,i} = 1$ when $t$ is in season i and 0 otherwise. Then for season 0, we
  have that $d_{t, i} - d_{t, 0} = -1$ and $d_{t, i} - d_{t, 0} = 1$ otherwise.
\end{itemize}
An important observation is that $\sum_{t=t_{j-1}^{\#}+1}^{t_j^{\#}}S_t = 0$.
The model is build this way in order to avoid breakpoints in
trend being caused by seasonal breaks. \\\\
In matrix form, the dummy seasonal model can be expressed as:
\[
X =
\begin{bmatrix}
 1 & 0 & 0 & \hdots & 0\\
 0 & 1 & 0 & \hdots & 0 \\
 0 & 0 & 1 & \hdots & 0 \\
\vdots & \vdots & \vdots & \ddots & \vdots \\
 0 & 0 & 0 & \hdots & 1 \\
 -1 & -1 & -1 & \hdots & -1 
\end{bmatrix}
\]
where $X$ has $s-1$ columns and $s$ rows.

\subsection{Harmonic Model}
Alternatively, we can use the Harmonic model that was mentioned in the context
of BFAST0n. For more information, please refer to Chapter \ref{chap:bfast0n}.

\section{Steps of the Algorithm}
\label{sec:bfast_algorithm_steps}
BFAST operates in a following way:
\begin{enumerate}
\item Estimate $\hat{S}_t$, using STL and finding the average of all seasons, by
  setting $n_s =$ ``harmonic''. For more information about STL parameters,
  please refer to Chapter \ref{chap:stl}.
\item Then iterate:
  \begin{enumerate}[1)]
    \item Calculate the deasonalized time series: $V_t = Y_t - \hat{S}_t$
    \item Apply the OLS-MOSUM test (Chapter \ref{chap:mosum}) to $V_t$. If the
      returned p-value is lower than the significance level $\alpha$, estimate
      the number and position of the trend components using the breakpoint
      estimation algorithm by Bai and Perron (Chapter \ref{chap:breakpoints}) from $V_t$.
    \item Compute the trend coefficients $\alpha_j$ and $\beta_j$ for
      $j = 1, \hdots, m$ using linear regression. Set the trend estimate
      $\hat{T}_t = \hat{\alpha}_j + \hat{\beta}_j t$ for
      $t = t^*_{j-1} + 1, \hdots, t^*_j$.
    \item Calculate the detrended time series: $W_t = Y_t - \hat{T}_t$
    \item If the OLS-MOSUM test, applied to $W_t$ signifies that the breakpoints
      are present in the seasonal data, estimate
      the number and position of the trend components using the breakpoint
      estimation algorithm by Bai and Perron (Chapter \ref{chap:breakpoints}) from $W_t$.
    \item Compute the coefficients for the seasonal component $\gamma_{i,j}$ for
      $j = 1, \hdots, m$ and $i = 1,\hdots s-1$. And reconstruct $\hat{S}_t$
      from the selected seasonal model and the seasonal coefficients.
  \end{enumerate}
  The iteration is stopped when the number and position of the breakpoints do
  not change or when we reach the maximum number of iterations.
\item Return $\hat{T}_t$, $\hat{S}_t$ the number and position of trend and
  seasonal breakpoints and the maximum breakpoint magnitude.
\end{enumerate}

\section{Parameters and Their Values}
\label{sec:bfast_params}
BFAST algorithm has following paramers
\begin{itemize}
\item $s$. Period of seasonality.
\item $0<h<1$. Minimal segment length for the Breakpoint estimation and
  bandwidth parameter for the OLS-MOSUM test. For more information please refer
  to Chapters \ref{chap:mosum} and \ref{chap:breakpoints} respectively.
  The default value is $0.15$.
\item \texttt{seasonal}. Type of model to be fitted. Should be either
  ``harmonic'', ``dummy'' or ``none''. The later means that no seasonal model
  would be fitted, i.e. $S_t=0$.
\item \texttt{max\_iter}. Maximal number of iterations of the main loop. The
  default value is $10$. In practice 1 or 2 iterations is normally enough.
\item \texttt{max\_breaks}. Upper limit on the number of breakpoints that would be found by
  the breakpoint estimation algorithm. The default value is ``None'', i.e. no upper limit.
  For more information, please refer to Chapter \ref{chap:breakpoints}.
\item $\alpha$. The confidence interval for the OLS-MOSUM test. If the returned
  p value is below that threshold - we consider that there is structural change
  and proceed with the linear regression. For more information, please refer to
  the Chapter \ref{chap:breakpoints}. The default value is $0.05$. 
\end{itemize}

\biblio
\end{document}
